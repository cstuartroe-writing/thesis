\chapter{Data}

\section{Structured paradigm data}

The English Wiktionary, a collaborative online dictionary, has become something of a standard source of supervised morphological data. \cite{Durrett2013} published a Wiktionary dataset of five paradigms from three languages (Finnish, Spanish, and German) which have been used as a benchmark in much future work. \parencite{Hulden2014} \parencite{Nicolai2015} \parencite{Ahlberg2015} \parencite{Faruqui2015} SIGMORPHON 2017 published \href{https://github.com/sigmorphon/conll2017/tree/master/all}{a dataset} partitioned into three training levels (100, 1000, and 10,000 tables), containing both sparse and full inflection tables for one or more parts of speech for each of 52 languages. Most of that data was derived from a January 2017 Wiktionary dump; data for four languages came from the Alexina project and data for Haida was prepared by Jordan Lachler. \parencite{Cotterell2017} 

The most complete structured dataset to date was published for the SIGMORPHON shared task 2018, a superset of the SIGMORPHON 2017 data, which is partitioned in the same manner and includes data from 103 languages. For most of the languages, data was scraped from Wiktionary.\

\section{Text corpora}

\section{Representation of morphology}

In earlier work, morphology is encoded in a language-specific and model-specific way.

\parencite{Luong2013} The UniMorph project, initially published in 2015, makes an effort to encode morphological categories uniformly cross-linguistically. \parencite{SylakGlassman2015} \parencite{SylakGlassman2015a} \parencite{SylakGlassman2016} The SIGMORPHON data since 2018 has been published in UniMorph format.